\subsection{Étude des Processus Métier}
Les processus métier constituent le mécanisme principal par lequel les services d'entreprise sont intégrés. C'est un ensemble d'activité visant à atteindre un objectif particulier d'une entreprise. Ce processus métier apporte une vision du métier réel, et constitue un excellent instrument de formalisation et d'analyse dans la construction des systèmes.
Dans le cas d'Oqenyite, les processus métier sont les suivants créer boutique, gérer catalogue en ligne, effectuer commande.
 

\subsubsection* {Gérer boutique}
Le processus gérer un boutique consiste à  
modifier les différents produits de la boutique: ajouter supprimer classer les produits par catégorie.  Après ça, il a la possibilité d’exposer ces produits, de donner ces caractéristiques et afficher les prix selon la catégorie de chaque produit. 

\subsubsection*{Gérer catalogue}

Ici le vendeur expose ces produits par catégorie en précisant les caractéristiques suivi des détails tout en mentionnant le prix de chaque produits, les mets en ligne.


\subsubsection*{Effectuer commande}

Pour faire des achats ou pour passer une commande le visiteur ou l'acheteur avant de voir les produit disponible sur la plate-forme d'oqenyite doit se connecter c'est à dire avoir un compte. Une fois connecté l'acheteur a la possibilité de consulter tous les produit existant sur la plate forme, voir leurs caractéristique, ainsi que le prix de chaque produits. Maintenant il fait le choix des produits désirés les ajoutent au panier et passe sa commande.



\subsubsection*{Gérer livraison}

La gestion de la livraison se fait comme suit, l'acheteur après avoir validation du panier choisir son adresse de livraison, si il n'a pas adresse de livraison, il a la possibilité d'ajouter son adresse. Il choisit ensuite le mode de livraison entre livraison (express ou classique). Il sélectionne un moyen de paiement et effectue le paiement.

Le processus métier nous conduit à la réalisation des digrammes d'activités et d'objet de flux.

