\section{Environnement économique}

 Le commerce électronique est l'un des facteurs phares pour le développement de l'économie. En effet, avec l'apparition du commerce électronique les relations entres vendeurs et acheteurs ont connu de grandes changement. L'e-commerce a surmonté l'handicape structurel de la distanciation physique et de prestation différée.
 Il a un atout important sur l'environnement économique, les ménages jouissent d'une nouvelle liberté, en pratiquant désormais ce commerce qui offre de nombreux services publics en ligne. Ce commerce poursuit une croissance d'extension vers la communication multimédia.
 
 L'e-commerce est une activité qui ne fait que croître depuis 2010. En 2017,
 les ventes en ligne sur les plates-formes d'e-commerce se sont élevées à
 2.304 milliards de dollars américain.
 Les Statistiques révèlent également qu'en 2017, les ventes en ligne en moyen-orient et en Afrique s'élèvent à seulement 16.651 millions de dollars américain, soit moins de 1\textdiscount des ventes mondiales. Ces statistiques sont énormes et montrent à quel point l'Afrique est en traîne dans ce domaine. Ce faible pourcentage est essentiellement dû à l'absence de moyens de paiement sécurisés. Cependant, avec la multiplication des moyens de paiement mobiles, l'Afrique peut espérer une croissance en e-commerce si toutefois des plates-formes d'e-commerce sécurisées, intégrant ces nouveaux moyens de paiement, voyaient le jour.
 
 
 
 
 
 