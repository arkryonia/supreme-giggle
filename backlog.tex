\subsection{Backlog}

Le Backlog Sprint est l’ensemble des éléments sélectionnés pour le Sprint plus un plan pour livrer l’incrément du produit et réaliser l’objectif du Sprint. Le Backlog Sprint est une prévision que l’équipe de développement fait de la fonctionnalité qui sera présente dans le prochain incrément et le travail nécessaire pour livrer cette fonctionnalité dans un incrément 
«Fini»
Le Backlog Sprint rend visible tout le travail que l'équipe de développement identifie comme nécessaire pour atteindre l'objectif du Sprint.
Le Backlog Sprint est un plan suffisamment détaillé pour que la progression soit compréhensible lors de la mêlée quotidienne.



%Un "backlog" est une liste de fonctionnalités ou de tâches, jugées nécessaires et suffisantes pour la réalisation satisfaisante du projet:
%Le backlog scrum est destiné à recueillir tous les besoins du client que l’équipe projet doit réaliser. Il contient donc la liste des fonctionnalités intervenant dans la constitution d’un produit, ainsi que tous les éléments nécessitant l’intervention de l’équipe projet. Tous les éléments inclus dans le backlog scrum sont classés par priorité indiquant l’ordre de leur réalisation.