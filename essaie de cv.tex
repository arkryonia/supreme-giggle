\documentclass[11pt,a4paper,sans,french]{moderncv}        % autres options possibles : taille de fonte ('10pt', '11pt' et '12pt'), format de papier ('a4paper', 'letterpaper', 'a5paper', 'legalpaper', 'executivepaper' and 'landscape') et famille de fonte ('sans' and 'roman')
\moderncvstyle{banking}                             % autres styles : 'casual' (défaut), 'classic', 'oldstyle' and 'banking'
\moderncvcolor{red}                               % autres couleurs : 'blue' (défaut), 'orange', 'green', 'red', 'purple', 'grey' and 'black'
%\nopagenumbers{}                                  % décommenter pour supprimer la numérotation automatique des pages pour les CVs de plus d'une page
\usepackage[utf8]{inputenc}                       % si vous n'utilisez pas xelatex ou lualatex, remplacer par le codage d'entrée que vous utilisez
\usepackage[scale=0.75,a4paper]{geometry}
\usepackage{babel}
%----------------------------------------------------------------------------------
%            informations personnelles
%----------------------------------------------------------------------------------
\firstname{Sourou}
\familyname{AVIMADJE}
%\title{Titre du CV}                               % optionnel : supprimer ou commenter si non souhaité
\address{BP}{Cotonou}{Bénin} % optionnel : supprimer ou commenter si non souhaité; l'argument « Pays » peut être omis ou vide
\mobile{66 76 72 90}                          % optionnel : supprimer ou commenter si non souhaité
%\phone{}                           % optionnel : supprimer ou commenter si non souhaité
%\fax{}                             % optionnel : supprimer ou commenter si non souhaité
\email{avimadjes@gmail.com}                               % optionnel : supprimer ou commenter si non souhaité
%\homepage{}                         % optionnel : supprimer ou commenter si non souhaité
%\extrainfo{}                 % optionnel : supprimer ou commenter si non souhaité
% \photo[64pt][0.4pt]{Image} % optionnel : décommenter si souhaité ; '64pt' est un exemple de hauteur que doit avoir la photo, 0.4pt est un exemple d'épaisseur que doit avoir le cadre qui l'entoure (à mettre à 0pt pour supprimer le cadre) et « Image » est le nom du fichier de la photo
%\quote{}                                 % optionnel : supprimer ou commenter si non souhaité
%
\begin{document}
\makecvtitle
\section{Formation}
\cventry{2008--2011}{Licence}{UATM}{Cotonou}{Assez-Bien}{}                      % les arguments 3 à 6 peuvent être laissés vides


\cventry{2005--2008}{BAC}{CEG Davie}{Porto-Novo}{}{}


\section{Expérience}
%\subsection{Principale}
\cventry{2004--2005}{Chargé des ventes}{Oxygen Broadband Network}{Lagos}{}{Chargé des ventes des produits informatique, Responsable de la clientèle à l'agence.}

\cventry{2012}{Agent vérificateur}{PADME-Bénin}{Porto-Novo}{}{Identification du domicile du lieu de vente et des garanties des clients, Réalisation d'enquête de moralitée sur les clients. Donne son avis sur l'octroi de crédit aux clients.}

\cventry{2011}{Agent de bureau}{PADME-Bénin}{Cotonou}{}{Formation des clients sur les produits du PADME.}
%\subsection{Divers}
%\cventry{Année--Année}{Emploi}{Employeur}{Ville}{}{Description générale d'au plus 1 ou 2 lignes}
\section{Langues}
\cvitemwithcomment{Français}{Excellente }{}
\cvitemwithcomment{Anglais}{Excellente}{}
\cvitemwithcomment{Fon}{Excellente}{}


\section{Compétences informatiques}
\cvdoubleitem{Word}{Excellent}{}{}
\cvdoubleitem{Excel}{Bien}{}{}
\cvdoubleitem{PowerPoint}{Passable}{}{}


\section{Centres d'intérêt}
\cvitem{La lecture}  {}
\cvitem{Le voyage}  {}
\cvitem{Le sport}   {}
%\section{Extra 1}
%\cvlistitem{Item 1}
%\cvlistitem{Item 2}
%\cvlistitem{Item 3}
%\section{Extra 2}
%\cvlistdoubleitem{Item 1}{Item 4}
%\cvlistdoubleitem{Item 2}{Item 5}
%\cvlistdoubleitem{Item 3}{Item 6}
%\section{References}
%\begin{cvcolumns}
%  \cvcolumn{Catégorie 1}{Commentaire}
%  \cvcolumn{Catégorie 2}{Commentaire}
%  \cvcolumn{Catégorie 3}{Commentaire}
%\end{cvcolumns}
%\clearpage
%%-----       letter       ---------------------------------------------------------
%% recipient data
%\recipient{DRH de l'entreprise}{UBA\\Numéro et rue\\Code postal et ville}
%\date{19/06/2018}
%\opening{cher monsieur,}
%\closing{Veuillez agréer,}
%\enclosure{Pièces jointes}          %  utiliser l'argument optionnel pour spécifier un autre mot que "Enclosure", ou redéfinir \enclname
%\makelettertitle
%\makeletterclosing
\end{document}
