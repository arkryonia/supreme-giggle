\section{Problématique}

Depuis l’avènement du commerce électronique, de nombreuses plate-formes offrant le service de vente en ligne
ont vu le jour. La multiplication de ces plates-formes a complètement changé
l’habitude de l’internaute : Depuis son ordinateur, sa tablette ou son
Smartphone, ce dernier peut acheter en ligne les produits qu’il désire sur la plate-forme qui lui convient.
En France cette activité à augmenter de manière très importante tant en terme de chiffre d'affaires plus de 20Milliards d'euro en 2012 avec un taux d'accroissement annuel de 20\textdiscount. Vu l'importance de ce commerce dans certains pays, il est regrettable de constater que cette activité reste peu développée en Afrique et particulièrement inexistantes au Bénin. Les quelques plates-formes ayant vu le jour, sont réalisées à base de solutions pré-conçues, solutions qui généralement, ne répondent pas aux exigences et besoins du marché local à cause des solutions de paiement qu'elles propose. Ainsi, pour palier ce problème et faire de l’e-commerce une activité effective au Bénin, il est impératif de développer des plate-formes de vente en ligne modernes et flexibles aux exigences du marché local et qui respectent les normes.