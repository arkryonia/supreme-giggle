\subsection{Méthode Agile Scrum}

 Cette méthode agile  permet la réalisation d'un projets complexe en favorisant l'interaction avec les membres de l'équipe et les managers, la collaboration du client et la réactivité face aux changement. Elle permet de produire une plus grande 
 valeur ajoutée dans la durée la plus  courte. Elle est une approche itérative et incrémentable, qui est menée dans un esprit collaboratif. Elle génère un produit de haute qualité tout en prenant en compte l’évolution des besoins des clients.


Scrum est la méthode Agile la plus utilisée de nos jours. En bref, elle définit des rôles: le Scrum Master, le Product Owner et l’équipe de développement, dicte la réitération de sprints, de production à durée limitée à la fin desquels des incréments fonctionnels de logiciel sont livrés et met en place des artefacts (le carnet de produit, le carnet de sprint, les graphiques d’avancement) ainsi que des cérémonies (planification de sprint, mêlée quotidienne, revue et rétrospective).

Elle implique l’auto-organisation des équipes et permet beaucoup plus la réactivité pour s’adapter aux besoins (parfois changeants) du client. Elle sous-entend aussi l’application de principes Agiles, soit la transparence, la simplicité et la collaboration.
La méthode Scrum soutient la livraison rapide et régulière de fonctionnalités à haute valeur ajoutée.



\subsection{Backlog}

Le Backlog Sprint est l’ensemble des éléments sélectionnés pour le Sprint plus un plan pour livrer l’incrément du produit et réaliser l’objectif du Sprint. Le Backlog Sprint est une prévision que l’équipe de développement fait de la fonctionnalité qui sera présente dans le prochain incrément et le travail nécessaire pour livrer cette fonctionnalité dans un incrément 
«Fini»
Le Backlog Sprint rend visible tout le travail que l'équipe de développement identifie comme nécessaire pour atteindre l'objectif du Sprint.
Le Backlog Sprint est un plan suffisamment détaillé pour que la progression soit compréhensible lors de la mêlée quotidienne.



%Un "backlog" est une liste de fonctionnalités ou de tâches, jugées nécessaires et suffisantes pour la réalisation satisfaisante du projet:
%Le backlog scrum est destiné à recueillir tous les besoins du client que l’équipe projet doit réaliser. Il contient donc la liste des fonctionnalités intervenant dans la constitution d’un produit, ainsi que tous les éléments nécessitant l’intervention de l’équipe projet. Tous les éléments inclus dans le backlog scrum sont classés par priorité indiquant l’ordre de leur réalisation.

\subsection{Items}


Les items d'un backlog sont les différents éléments constitutifs d'un backlog produit.Le Backlog Produit est une liste ordonnée de tous les éléments identifiés comme nécessaires au produit. Il constitue l’unique source d'exigences pour tout changement à apporter au produit. Le Backlog Produit liste toutes les fonctionnalités, les fonctions, les exigences, les améliorations et les corrections qui constituent des modifications à apporter au produit dans les versions futures. Les éléments du backlog produit se composent d'une description, d'un ordre, d'une
estimation et d'une valeur. Les éléments du backlog produit incluent souvent des descriptions du test qui prouveront leur complétude lorsqu’ils sont «Finis ».


\subsection{Sprint}

 Un Sprint est défini pour réaliser un objectif, la définition des fonctionnalités de l’activité à développer, la conception et le plan flexible qui guidera le développement, la durée du sprint est limitée à  (moins d’un mois).
 Il contient et est constitué de la planification du Sprint, des mêlées quotidienne, des activités de développement, de la revue du Sprint et de la rétrospective du Sprint.
 Le sprint a un objectif fixe auquel est associée une liste d’éléments du Product backlog, ce but est sans changements qui le remettent en cause. Les objectifs de qualités sont maintenus.
 Sprints amènent de la prévisibilité en forçant une inspection et adaptation du progrès vers l’atteinte d’un objectif au moins mensuellement.

\subsection{La Mêlée}

La mêlée quotidienne, encore appelé daily scrum est un événement limité à 15 minutes au cours duquel l'équipe de développement synchronise ses activités et crée un plan pour les prochaine heures.
Elle réunit tous les membres de l’équipe et permet d’examiner les tâches en cours et les difficultés rencontrées.
Les mêlées quotidiennes améliorent la communication, éliminent les autres réunions, identifient
les obstacles qui perturbent le développement afin qu'ils soient éliminés, mettent en avant et encouragent la prise de décision rapide tout en améliorant le niveau de connaissance au sein de l’équipe de
développement. Il s’agit d’un point clé d’inspection et d’adaptation.

%C'est ce qui permet de mettre au quotidien l’application des principes inspection-adaptation de la méthode Scrum.
%
%La mêlée quotidienne est une réunion interne pour l'équipe de développement. 
%Si d'autres personnes sont présentes, le Scrum Master s'assure qu'elles ne perturbent pas la réunion.

