\chapter{Cadre contextuel}
\section{Problématique}
Depuis l'avènement du commerce électronique, de nombreuses plate-formes offrant le service de vente en ligne ont vu le jour. La multiplication de ces plates-formes a complètement changé le comportement de l’internaute : depuis son ordinateur, sa tablette ou son Smartphone, ce dernier peut acheter en ligne les produits qu'il désire sur la plate-forme qui lui convient. 

En France cette activité à augmenter de manière très importante en terme de chiffre d'affaires plus de 20 Milliards d'euro en 2012 avec un taux d'accroissement annuel de 20 \%. Vu l'importance de ce commerce dans les pays du nord, il est regrettable de constater que cette activité reste peu développée en Afrique et particulièrement inexistantes au Bénin. Les quelques plates-formes qui y ont vu le jour, sont réalisées à base de solutions pré-conçues, solutions qui généralement, ne répondent pas aux exigences et besoins du marché local à cause des solutions de paiement qu'elles proposent. 

Ainsi, pour palier à ce problème et faire de l’e-commerce une activité effective au Bénin, il est impératif de développer des plate-formes de vente en ligne modernes et flexibles qui répondent aux exigences du marché local et qui respectent les normes sécuritaires.

\section{Environnement économique global}
Le commerce électronique est l'un des facteurs phares pour le développement de l'économie. En effet, avec l'apparition du commerce électronique les relations entres vendeurs et acheteurs ont connu de grandes changement. L'e-commerce a surmonté l'handicape structurel de la distanciation physique et de prestation différée.
Il a un atout important sur l'environnement économique, les ménages jouissent d'une nouvelle liberté, en pratiquant désormais ce commerce qui offre de nombreux services publics en ligne. Ce commerce poursuit une croissance d'extension vers la communication multimédia.

L'e-commerce est une activité qui ne fait que croître depuis 2010. En 2017, les ventes en ligne sur les plates-formes d'e-commerce se sont élevées à 2.304 milliards de dollars américain.

\section{Environnement économique local}
En 2017, les ventes en ligne en moyen-orient et en Afrique s'élèvent à seulement 16.651 millions de dollars américain, soit moins de 1\% des ventes mondiales. Ces statistiques sont énormes et montrent à quel point l'Afrique est en traîne dans ce domaine. Ce faible pourcentage est essentiellement dû à l'absence de moyens de paiement sécurisés. Cependant, avec la multiplication des moyens de paiement mobiles, l'Afrique peut espérer une croissance en e-commerce si toutefois des plates-formes d'e-commerce sécurisées, intégrant ces nouveaux moyens de paiement, voyaient le jour.
