\section{Concepts clés}
\subsection{Le commerce électronique ou e-commerce}
Le commerce électronique regroupe l’ensemble des transactions commerciales s’opérant à distance par le biais d’interfaces électroniques et digitales à partir des différents types de terminaux (Ordinateurs, tablettes, smartphones, consoles, TV connectées).

\subsection{Le Marché virtuel}
Le marché virtuel est un marché qui le plus souvent à une dimension internationale et qui sont plutôt des évènements permanents. Il permet aux investisseurs de se présenter, d’entrer en contact facilement, de faire connaitre leurs besoins en matière de développement et de trouver des partenaires pour satisfaire ces besoins.

\subsection{E-Vendeur}
E-vendeur est un vendeur en ligne, il est chargé de rendre disponible ses produits, donner ces caractéristiques et d'ajouter le prix de chacun de ces articles dans le but de les vendre aux clients actuels.

\subsection{E-Acheteur}
Les e-acheteurs sont fortement demandeurs d’offres personnalisées, notamment parmi les adhérents à des programmes de fidélisation. L'e-acheteur à un rôle très important. En effet, il se charge de consulter les produits en ligne, sélection, paie, et reçoit en toute sécurité les produits commander.

\subsection{E-Boutique}
E-boutique est une boutique de vente de produits, des biens et services en ligne. Grâce à une boutique en ligne, on peut choisir et payer des articles comme dans un magasin réel. Pour acheter un produit dans cette boutique virtuelle, il suffit de choisir les produits désirés puis de les mettre dans un panier. L’acheteur peut remplir un bon et payer sa commande par carte bancaire ou par un autre moyen de paiement. La commande sera livrée en fonction du choix de l’internaute et selon les modalités définies par le responsable de la boutique.


\section{Approche et démarche d'analyse}

Les auteurs d’UML préconisent l’utilisation d'une démarche itérative, incrémentale et guidée par les besoins des utilisateurs du'un système dans la réalisation d’une application informatique.

La méthode Agile Scrum est la méthode utilisée pour cette réalisation. Cette méthode de réalisation du projet est basée sur des indications. Avec cette méthode, un processus est défini et suivi pour la réalisation.


\subsection{Méthode Agile Scrum}
Cette méthode agile  permet la réalisation d'un projets complexe en favorisant l'interaction avec les membres de l'équipe et les managers, la collaboration du client et la réactivité face aux changement. Elle permet de produire une plus grande 
valeur ajoutée dans la durée la plus  courte. Elle est une approche itérative et incrémentable, qui est menée dans un esprit collaboratif. Elle génère un produit de haute qualité tout en prenant en compte l’évolution des besoins des clients.


Scrum est la méthode Agile la plus utilisée de nos jours. En bref, elle définit des rôles: le Scrum Master, le Product Owner et l’équipe de développement, dicte la réitération de sprints, de production à durée limitée à la fin desquels des incréments fonctionnels de logiciel sont livrés et met en place des artefacts (le carnet de produit, le carnet de sprint, les graphiques d’avancement) ainsi que des cérémonies (planification de sprint, mêlée quotidienne, revue et rétrospective).

Elle implique l’auto-organisation des équipes et permet beaucoup plus la réactivité pour s’adapter aux besoins (parfois changeants) du client. Elle sous-entend aussi l’application de principes Agiles, soit la transparence, la simplicité et la collaboration.
La méthode Scrum soutient la livraison rapide et régulière de fonctionnalités à haute valeur ajoutée.

\subsubsection{Backlog}
Le Backlog Sprint est l’ensemble des éléments sélectionnés pour le Sprint plus un plan pour livrer l’incrément du produit et réaliser l’objectif du Sprint. Le Backlog Sprint est une prévision que l’équipe de développement fait de la fonctionnalité qui sera présente dans le prochain incrément et le travail nécessaire pour livrer cette fonctionnalité dans un incrément «Fini»
Le Backlog Sprint rend visible tout le travail que l'équipe de développement identifie comme nécessaire pour atteindre l'objectif du Sprint.
Le Backlog Sprint est un plan suffisamment détaillé pour que la progression soit compréhensible lors de la mêlée quotidienne.

\subsubsection{Items}
Les items d'un backlog sont les différents éléments constitutifs d'un backlog produit.Le Backlog Produit est une liste ordonnée de tous les éléments identifiés comme nécessaires au produit. Il constitue l’unique source d'exigences pour tout changement à apporter au produit. Le Backlog Produit liste toutes les fonctionnalités, les fonctions, les exigences, les améliorations et les corrections qui constituent des modifications à apporter au produit dans les versions futures. Les éléments du backlog produit se composent d'une description, d'un ordre, d'une
estimation et d'une valeur. Les éléments du backlog produit incluent souvent des descriptions du test qui prouveront leur complétude lorsqu’ils sont «Finis ».

\subsubsection{Sprint}
Un Sprint est défini pour réaliser un objectif, la définition des fonctionnalités de l’activité à développer, la conception et le plan flexible qui guidera le développement, la durée du sprint est limitée à  (moins d’un mois).
Il contient et est constitué de la planification du Sprint, des mêlées quotidienne, des activités de développement, de la revue du Sprint et de la rétrospective du Sprint.
Le sprint a un objectif fixe auquel est associée une liste d’éléments du Product backlog, ce but est sans changements qui le remettent en cause. Les objectifs de qualités sont maintenus.
Sprints amènent de la prévisibilité en forçant une inspection et adaptation du progrès vers l’atteinte d’un objectif au moins mensuellement.

\subsubsection{La Mêlée}
La mêlée quotidienne, encore appelé daily scrum est un événement limité à 15 minutes au cours duquel l'équipe de développement synchronise ses activités et crée un plan pour les prochaine heures.
Elle réunit tous les membres de l’équipe et permet d’examiner les tâches en cours et les difficultés rencontrées.
Les mêlées quotidiennes améliorent la communication, éliminent les autres réunions, identifient
les obstacles qui perturbent le développement afin qu'ils soient éliminés, mettent en avant et encouragent la prise de décision rapide tout en améliorant le niveau de connaissance au sein de l’équipe de développement. Il s’agit d’un point clé d’inspection et d’adaptation.
