\subsection{Sprint}

 Un Sprint est défini pour réaliser un objectif, la définition des fonctionnalités de l’activité à développer, la conception et le plan flexible qui guidera le développement, la durée du sprint est limitée à  (moins d’un mois).
 Il contient et est constitué de la planification du Sprint, des mêlées quotidienne, des activités de développement, de la revue du Sprint et de la rétrospective du Sprint.
 Le sprint a un objectif fixe auquel est associée une liste d’éléments du Product backlog, ce but est sans changements qui le remettent en cause. Les objectifs de qualités sont maintenus.
 Sprints amènent de la prévisibilité en forçant une inspection et adaptation du progrès vers l’atteinte d’un objectif au moins mensuellement.